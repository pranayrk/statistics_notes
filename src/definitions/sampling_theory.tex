\section{Sampling Theory}
We can either sample \textit{with replacement} or \textit{without replacement}. A finite population sampled with replacement can be considered infinite. Sampling from a very large finite population can similarly be considered as sampling from an infinite population. 

To properly choose the sample, we can make sure that every member of the population has an equal chance of being in the sample.
Normally, since the sample size is much smaller than the population size, sampling without replacement will give practically the same results as sampling with replacement.

For a sample of size $n$ from a population which we assume has distribution $f(x)$, we can choose members of the population at random, each selection corresponding to a random variable $X_1, X_2, ..., X_n$ with corresponding values $x_1, x_2, ..., x_n$. In case we are assuming sampling without replacement, $X_1, X_2, ..., X_n$ will be independent and identically distributed random variables with probability distribution $f(x)$.

\begin{defn}
Let $X$ be a random variable with a distribution $f$, and let $X_1, X_2, ..., X_n$ be iid random variables with the common distribution $f$.

Then the collection $X_1, X_2, ..., X_n$ is called a \textbf{random sample} of size $n$ from the population $f$.
\end{defn}

Since $X_1, X_2, ..., X_n$ are iid, the joint distribution of the random sample is $f(x_1, x_2, ..., x_n) = f(x_1) f(x_2) ... f(x_n)$.

Any quantity obtained from a sample for the purpose of estimating a population parameter is called a sample
statistic, or briefly statistic. Mathematically, a sample statistic for a sample of size n can be defined as a function of the random variables $X_1, X_2, ..., X_n$ as $T(X_1, X_2, ..., X_n)$. This itself is a random variable whose values can be represented as $T(x_1, x_2, ..., x_n)$.

\begin{defn}
    Let $X_1, X_2, ..., X_n$ be a random sample of size $n$ from the population whose distribution is $f\left (x|\theta\right )$ (the distribution $f$ with unknown parameter $\theta$). Let $T\left (x_1, x_2, ...,x_n\right )$ be a real-valued or vector-valued function whose domain includes the range of $\left (X_1, X_2, ..., X_n\right )$. Then the random variable or random vector $Y = T\left (X_1, ..., X_n\right )$ is called a \textbf{statistic} provided that $T$ is not a function of any unknown parameter $\theta$.
\end{defn}

For example consider $X \approx N(\mu, \sigma^2)$ where $\mu$ is known but $\sigma$ is unknown. Then $\frac{\sum_{i=1}^n X_i}{\sigma^2}$ is not a statistic but $\frac{\sum_{i=1}^n X_i}{\mu^2}$ is a statistic.

Two common statistics are the sample mean and sample variance.

\begin{defn}
    The \textbf{sample mean} is the arithmetic average of the values in the random sample. It is denoted by $\bar{X} = \displaystyle \frac{X_1 + X_2 + ... + X_n}{n}$.

    The \textbf{sample variance} is the statistic defined by $S^2 = \displaystyle \frac{\sum_{i=1}^n \left (X_i - \bar{X}\right )^2}{n-1}$

    The \textbf{sample standard deviation} is the statistic defined by $S = \sqrt{S^2}$.
\end{defn}

\begin{defn}
    Let $X_1, X_2, ..., X_n$ be a random sample from a population $f\left (x| \theta\right )$. We say that a statistic $T\left (X_1, X_2, ...,X_n\right )$ is an \textbf{unbiased estimator} of the parameter $\theta$ if $E(T) = \theta$ for all possible values of $\theta$.
\end{defn}

\begin{thm}
    Let $X_1, X_2, ..., X_n$ be a random sample from a population with mean $\mu $ and variance $\sigma ^2 < \infty$. Then:
    \begin{enumerate}
        \item $E\left (\bar{X}\right ) = \mu $
        \item $\text{Var}\left (\bar{X}\right ) = \frac{\sigma ^2}{n}$
        \item $E\left (S ^2\right ) = \sigma ^2$
    \end{enumerate}
\end{thm}

From the above theorem we see that the sample mean $\bar{X}$ is an unbiased estimator of the population mean $\mu$ and the sample variance $S^2$ is an unbiased estimator of the population variance $\sigma^2$. (The reason we included $1/n-1$ in the  definition of the sample variance was to make it an unbiased estimator)

\begin{defn}
    Let $X_1, X_2, ..., X_n$ be a random sample of size $n$ from a population $f(x|\theta)$. The probability distribution of a statistic $T(X_1, X_2, ..., X_n)$ is called the sampling distribution of $T$.
\end{defn}

\begin{thm}[Distribution of the sample mean]
    Let $X_1, X_2, ..., X_n$ be a random sample from a population with MGF $M_X(t)$. Then the MGF of the sample mean is $M_{\bar{X}} (t) = (M_X (t/n))^n$.
\end{thm}

\begin{thm}
    Let $X_1, X_2, . . . , X_n$ be a random sample from a $N(\mu, \sigma^2)$ distribution, and let $X$ denote the sample mean.

    Then $X$ and the random vector $(X_1 - X, X_2 - X, . . . , X_n - X)$ are independent.
\end{thm}

\begin{thm}
    Let $X_1, X_2, . . . , X_n$ be a random sample from a $N(\mu, \sigma^2)$ distribution, 
    and let $X$ denote the sample mean and $S^2$ denote the sample variance. Then $X$ and $S^2$ are independent random variables.
\end{thm}
The converse of this theorem is also true: if the sample mean and sample variance of a random sample are independent random variables then population distribution is normal.
