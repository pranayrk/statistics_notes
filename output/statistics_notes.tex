\documentclass[12pt,twoside]{report}
\usepackage[utf8]{inputenc}
\usepackage{amsmath,amsfonts,graphicx,amsthm}
\usepackage{epsfig,amssymb}
\usepackage{caption}
\usepackage{fancyhdr}
\usepackage{lipsum}
\usepackage{thmtools}

\newtheorem{defn}{Definition}
\newtheorem{note}{Note}
\newtheorem{notation}[subsection]{Notation}
\newtheorem{thm}[subsection]{Theorem}
\newtheorem{corollary}[subsection]{Corollary}
\newtheorem{eg}[subsection]{Example}
\newtheorem{lemma}[subsection]{Lemma}
\newtheorem{result}[subsection]{Result}
\declaretheoremstyle[bodyfont=\normalfont]{normalbody}
\declaretheorem[style=normalbody,name=]{ex}
\newtheorem{remark}[subsection]{Remark}
\newtheorem{prop}[subsection]{Proposition}
\newenvironment*{ans}{\textbf{ans.}\space\em\\}{\par}
\newenvironment*{source}{\hfill\scriptsize\textbf{Source.}\space}{\par}

\declaretheoremstyle[headfont=\normalfont]{normalhead}

\makeatletter
\newcommand*{\rom}[1]{\expandafter\@slowromancap\romannumeral #1@}
\makeatother

\widowpenalties 1 10000
\raggedbottom
\setlength{\parindent}{0pt}
\renewcommand{\chaptername}{}

\begin{document}

\tableofcontents
\newpage
\pagenumbering{arabic}

\chapter{Definitions and Theory}
\line(1,0){360} \\

We can either sample \textit{with replacement} or \textit{without replacement}. A finite population sampled with replacement can be considered infinite. Sampling from a very large finite population can similarly be considered as sampling from an infinite population. 

To properly choose the sample, we can make sure that every member of the population has an equal chance of being in the sample.
Normally, since the sample size is much smaller than the population size, sampling without replacement will give practically the same results as sampling with replacement.

For a sample of size $n$ from a population which we assume has distribution $f(x)$, we can choose members of the population at random, each selection corresponding to a random variable $X_1, X_2, ..., X_n$ with corresponding values $x_1, x_2, ..., x_n$. In case we are assuming sampling without replacement, $X_1, X_2, ..., X_n$ will be independent and identically distributed random variables with probability distribution $f(x)$.

\begin{defn}[Random Sample]
Let $X$ be a random variable with a distribution $f$, and let $X_1, X_2, ..., X_n$ be iid random variables with the common distribution $f$.

Then the collection $X_1, X_2, ..., X_n$ is called a \textbf{random sample} of size $n$ from the population $f$.
\end{defn}

Since $X_1, X_2, ..., X_n$ are iid, the joint distribution of the random sample is $f(x_1, x_2, ..., x_n) = f(x_1) f(x_2) ... f(x_n)$.

Any quantity obtained from a sample for the purpose of estimating a population parameter is called a sample
statistic, or briefly statistic. Mathematically, a sample statistic for a sample of size n can be defined as a function of the random variables $X_1, X_2, ..., X_n$ as $T(X_1, X_2, ..., X_n)$. This itself is a random variable whose values can be represented as $T(x_1, x_2, ..., x_n)$.

\begin{defn}[Statistic]
    Let $X_1, X_2, ..., X_n$ be a random sample of size $n$ from the population whose distribution is $f\left (x|\theta\right )$ (the distribution $f$ with unknown parameter $\theta$). Let $T\left (x_1, x_2, ...,x_n\right )$ be a real-valued or vector-valued function whose domain includes the range of $\left (X_1, X_2, ..., X_n\right )$. Then the random variable or random vector $Y = T\left (X_1, ..., X_n\right )$ is called a \textbf{statistic} provided that $T$ is not a function of any unknown parameter $\theta$.
\end{defn}

For example consider $X \approx N(\mu, \sigma^2)$ where $\mu$ is known but $\sigma$ is unknown. Then $\frac{\sum_{i=1}^n X_i}{\sigma^2}$ is not a statistic but $\frac{\sum_{i=1}^n X_i}{\mu^2}$ is a statistic.

Two common statistics are the sample mean and sample variance.

\begin{defn}[Sample Mean, Sample Variance, Sample Standard Deviation]
    The \textbf{sample mean} is the arithmetic average of the values in the random sample. It is denoted by $\bar{X} = \displaystyle \frac{X_1 + X_2 + ... + X_n}{n}$.

    The \textbf{sample variance} is the statistic defined by $S^2 = \displaystyle \frac{\sum_{i=1}^n \left (X_i - \bar{X}\right )^2}{n-1}$

    The \textbf{sample standard deviation} is the statistic defined by $S = \sqrt{S^2}$.
\end{defn}

\begin{defn}[Unbiased Estimator]
    Let $X_1, X_2, ..., X_n$ be a random sample from a population $f\left (x| \theta\right )$. We say that a statistic $T\left (X_1, X_2, ...,X_n\right )$ is an \textbf{unbiased estimator} of the parameter $\theta$ if $E_T = \theta$ for all possible values of $\theta$.
\end{defn}

\begin{thm}
    Let $X_1, X_2, ..., X_n$ be a random sample from a population with mean $\mu $ and variance $\sigma ^2 < \infty$. Then:
    \begin{enumerate}
        \item $E\left (\bar{X}\right ) = \mu $
        \item $\text{Var}\left (\bar{X}\right ) = \frac{\sigma ^2}{n}$
        \item $E\left (S ^2\right ) = \sigma ^2$
    \end{enumerate}
\end{thm}

From the above theorem we see that the sample mean $\bar{X}$ is an unbiased estimator of the population mean $\mu$ and the sample variance $\S^2$ is an unbiased estimator of the population variance $\sigma^2$. (The reason we included $1/n-1$ in the  definition of the sample variance was to make it an unbiased estimator)

\newpage
\chapter{Exercises}
\line(1,0){360} \\


\begin{ex}
Let $X_1, X_2, ..., X_n$ be a random sample of $n$ identical circuit boards whose times until failure are thought to follow an exponential($\beta$) population. Find the joint distribution of the sample. What is the probability that all the boards last more than 2 years?
\end{ex}
\begin{ans}
$f(x_1, x_2, ..., x_n)  = f(x_1) f(x_2) ... f(x_n) = \Pi_{i=1}^n \frac{1}{\beta} \exp(\frac{-x_i}{\beta})$ \\
$P(X_1 > 2, X_2 > 2, ..., X_n > 2) = \exp(\frac{-2n}{\beta})$
\end{ans}
\begin{source}
    Class, Lec 02
\end{source}

\begin{ex}
If sample $X_1, X_2, ..., X_n$ are drawn from  a finite population without replacement, then show that the random variables $X_1, X_2, ...,X_n$ are not mutually independent but that they are identically distributed.
\end{ex}
\begin{source}
Class, Lec 02 
\end{source}

\begin{samepage}
\begin{ex}
Let $X \approx \text{Bernoulli}\left (p\right )$ where $p$ is possibly unknown. Suppose that five independent observations on $X$ are $0,1,1,1,0$ Then find the sample mean, sample variance and sample standard deviation.
\end{ex}
\begin{ans}
Mean $\bar{x} = 0.6$\\
Variance $s^2 = 0.3$ \\
Standard Deviation $s = 0.55$ 
\end{ans}
\begin{source}
Class, Lec 03
\end{source}
\end{samepage}

\begin{samepage}
\begin{ex}
    Let $X_1, X_2, ..., X_n$ be a random sample from a population with mean $\mu $ and variance $\sigma ^2 < \infty$. Then show that:
    \begin{enumerate}
        \item $E\left (\bar{X}\right ) = \mu $
        \item $\text{Var}\left (\bar{X}\right ) = \frac{\sigma ^2}{n}$
        \item $E\left (S ^2\right ) = \sigma ^2$
    \end{enumerate}
\end{ex}
\begin{source}
Class, Lec 03
\end{source}
\end{samepage}

\begin{samepage}
\begin{ex}
Color blindness appears in 1\%  of the people in a certain population. How large must a sample be if the probability of its containing a color-blind person is to be 0.95 or more? (Assume that the population is large enough to be considered infinite, so that sampling can be considered to be with replacement.)
\end{ex}
\begin{ans}
Sample size greater than 299
\end{ans}
\begin{source}
Class, Lec 03
\end{source}
\end{samepage}

\begin{samepage}
\begin{ex}
Suppose that five observations on normal population are \\
$-0.864, 0.561, 2.355, 0.582, -0.774$.
Compute sample variance.
\end{ex}
\begin{ans}
1.648
\end{ans}
\begin{source}
Class, Lec 03
\end{source}
\end{samepage}

\newpage
\addcontentsline{toc}{chapter}{Bibliography}
\begin{thebibliography}{9}
    \bibitem{latexcompanion}
        (Schaum’s Outlines) John Schiller, R. Alu Srinivasan, Murray Spiegel - Probability and Statistics-McGraw-Hill Education (2012)
    \bibitem{latexcompanion}
        \lipsum[2][1-3]
    \bibitem{latexcompanion}
        \lipsum[3][1-3]
    \bibitem{latexcompanion}
        \lipsum[4][1-3]
    \bibitem{latexcompanion}
        \lipsum[5][1-3]
    \bibitem{latexcompanion}
        \lipsum[6][1-3]
\end{thebibliography}
\end{document}

