\documentclass[12pt,twoside]{report}
\usepackage[utf8]{inputenc}
\usepackage{amsmath,amsfonts,graphicx,amsthm}
\usepackage{epsfig,amssymb}
\usepackage{caption}
\usepackage{fancyhdr}
\usepackage{lipsum}
\usepackage{thmtools}

\theoremstyle{remark}

\newtheorem{defn}{Definition}
\newtheorem{note}{Note}
\newtheorem{notation}[subsection]{Notation}
\newtheorem{thm}[subsection]{Theorem}
\newtheorem{corollary}[subsection]{Corollary}
\newtheorem{eg}[subsection]{Example}
\newtheorem{lemma}[subsection]{Lemma}
\newtheorem{result}[subsection]{Result}
\declaretheoremstyle[bodyfont=\normalfont]{normalbody}
\declaretheorem[style=normalbody,name=Exercise]{ex}
\newtheorem{remark}[subsection]{Remark}
\newtheorem{prop}[subsection]{Proposition}
\newenvironment*{sol}{\textbf{Solution.}\space\em\\}{\par}
\newenvironment*{source}{\hfill\scriptsize\textbf{Source.}\space}{\par}

\declaretheoremstyle[headfont=\normalfont]{normalhead}

\makeatletter
\newcommand*{\rom}[1]{\expandafter\@slowromancap\romannumeral #1@}
\makeatother

\widowpenalties 1 10000
\raggedbottom
\setlength{\parindent}{0pt}

\begin{document}

\tableofcontents
\newpage
\pagenumbering{arabic}

\chapter{Definitions and Theory}
\line(1,0){360} \\

We can either sample \textit{with replacement} or \textit{without replacement}. A finite population sampled with replacement can be considered infinite. Sampling from a very large finite population can similarly be considered as sampling from an infinite population. 

To properly choose the sample, we can make sure that every member of the population has an equal chance of being in the sample.
Normally, since the sample size is much smaller than the population size, sampling without replacement will give practically the same results as sampling with replacement.

For a sample of size $n$ from a population which we assume has distribution $f(x)$, we can choose members of the population at random, each selection corresponding to a random variable $X_1, X_2, ..., X_n$ with corresponding values $x_1, x_2, ..., x_n$. In case we are assuming sampling without replacement, $X_1, X_2, ..., X_n$ will be independent and identically distributed random variables with probability distribution $f(x)$.

\begin{defn}[Random Sample]
Let $X$ be a random variable with a distribution $f$, and let $X_1, X_2, ..., X_n$ be iid random variables with the common distribution $f$.

Then the collection $X_1, X_2, ..., X_n$ is called a \textbf{random sample} of size $n$ from the population $f$.
\end{defn}

Since $X_1, X_2, ..., X_n$ are iid, the joint distribution of the random sample is $f(x_1, x_2, ..., x_n) = f(x_1) f(x_2) ... f(x_n)$.




\newpage
\chapter{Exercises}
\line(1,0){360} \\


\begin{ex}
Let $X_1, X_2, ..., X_n$ be a random sample of $n$ identical circuit boards whose times until failure are thought to follow an exponential($\beta$) population. Find the joint distribution of the sample. What is the probability that all the boards last more than 2 years?
\end{ex}
\begin{sol}
$f(x_1, x_2, ..., x_n)  = f(x_1) f(x_2) ... f(x_n) = \Pi_{i=1}^n \frac{1}{\beta} \exp(\frac{-x_i}{\beta})$ \\
$P(X_1 > 2, X_2 > 2, ..., X_n > 2) = \exp(\frac{-2n}{\beta})$
\end{sol}
\begin{source}
    Class, Lec 02
\end{source}

\newpage
\addcontentsline{toc}{chapter}{Bibliography}
\begin{thebibliography}{9}
    \bibitem{latexcompanion}
        (Schaum’s Outlines) John Schiller, R. Alu Srinivasan, Murray Spiegel - Probability and Statistics-McGraw-Hill Education (2012)
    \bibitem{latexcompanion}
        \lipsum[2][1-3]
    \bibitem{latexcompanion}
        \lipsum[3][1-3]
    \bibitem{latexcompanion}
        \lipsum[4][1-3]
    \bibitem{latexcompanion}
        \lipsum[5][1-3]
    \bibitem{latexcompanion}
        \lipsum[6][1-3]
\end{thebibliography}
\end{document}

